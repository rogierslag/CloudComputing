\documentclass[a4paper]{IEEEtran}

\usepackage[numbers,square,comma]{natbib}

\title{Cloud Computing: \\ Report}
\author{Steffan Norberhuis\\ 1509306 \and
 Rogier Slag\\ 1507761}

\author{
    \IEEEauthorblockN{Steffan Norberhuis, Rogier Slag}\\
    \IEEEauthorblockA{1509306, 1507761}
}

\begin{document}
\maketitle
\begin{center}
\today
\end{center}

\section{Abstract}


\section{Introduction}
%Recommended size abstract and introduction 1 page
%Describe the problem

%Related work

%To be implemented system

%structure of arcticle

\section{Background on Application}

The application has as goal to convert any user submitted video to an MPEG-4 format (note this format is covered by several patents in countries which acknowledge software algorithms patents \cite{mpegla}.
For this, users can use the existing client software of WantCloud to upload their video to the servers and receive a notification of the same application once the conversion is done.
In the background, transparent to the user, the video is added to the job list by the scheduler software running on the incoming server.
This scheduler, together with the part which handles provisioning of VM's on a IaaS-provider, places the job on a server for conversion.
Due to the nature of video-conversion, a server is considered busy if there is one job running on it, and idle if no job is running on it. 
No server will handle two jobs at once.
The provisioner may decide to pin up additional VM's based on several, user-defined, criteria.
Once the conversion is done, the worker VM uploads the new file to the output server, which then performs a callback on the earlier mentioned client application.\\
\\
\textit{Requirements:} For the system to work in production, it should ensure the following requirements are met:

\begin{enumerate}
\item For the system to work well for WantCloud B.V., it should have a better performance compared to the existing platform, while at the same time it should lower monthly operational costs,
\item The system should work completely automated, requiring no intervention for any worker nodes, this goes for the autoscaling, handle worker failures, and scheduling.
\item The system should monitor its own state, both in terms of performance as well as for availability.
\item The results should remain available for users to re-download, hence durability is important.

\end{enumerate}


\section{System Design}
%Recommended Size 1.5 pages

\subsection{Resource Management Architecture}
%Design

%Inter operation provisioning

%Allocation

%Reliability

%Monitoring components


\subsection{System Policies}


\section{Experimental Results}
%Recommended size 1.5 pages

\subsection{Experimental setup}
%Describe working environments

%Describe general workload

%Describe monitoring tools and libraries

%other tools

\subsection{Experiments}
%Desribe the experiments per system feature
%Format: describe workload, present operation, analyze result.
\subsubsection{Provisioning}

\subsubsection{Allocation}

\subsubsection{Reliability}

\subsubsection{Monitoring components}

%Report charged-time
%Report charged cost
%Report service metrics

\section{Discussion}

\section{Conclusion}

\bibliography{bibliography}
\bibliographystyle{unsrtnat}


\appendix
\section{Time Sheets}
\end{document}
