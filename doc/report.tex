\documentclass[a4paper]{IEEEtran}

\title{Cloud Computing: \\ Report}
\author{Steffan Norberhuis\\ 1509306 \and
 Rogier Slag\\ 1507761}

\author{
    \IEEEauthorblockN{Steffan Norberhuis, Rogier Slag}\\
    \IEEEauthorblockA{1509306, 1507761}
}

\begin{document}
\maketitle
\begin{center}
\today
\end{center}

\section{Abstract}


\section{Introduction}
%Recommended size abstract and introduction 1 page
%Describe the problem

%Related work

%To be implemented system

%structure of arcticle

\section{Background on Application}
%1-3 paragraph
%Describe application

%Requirements


\section{System Design}
%Recommended Size 1.5 pages

\subsection{Resource Management Architecture}
%Design

%Inter operation provisioning

%Allocation

%Reliability

%Monitoring components


\subsection{System Policies}


\section{Experimental Results}
%Recommended size 1.5 pages
In this section we will describe the experimental setup and the experiments themselves.
Observations and experiments are also included in this section.

\subsection{Experimental setup}
%Describe working environments
The system was developed for Amazon AWS and was tested on that platform.
Amazon AWS provides different types of instances and we used \emph{t2.micro} instances.
Our workload is mostly CPU intensive.
These are the smallest instances provided and we could achieve higher performance on larger instances,
 but they were chosen because they are Free Tier eligible.
Amazon describes these instances as CPU's operating at 2.5GHz and having 1 GB of memory.

%Describe general workload
We created a workload generator that generates an artificial workload.
The workload generator creates a new task at a regular interval and offers this task to the system using the input bucket.
The task can either be a completely new task or with a certain chance be a task already offered to the system before.
The tasks consist of encoding a movie from one format to another format.

%Describe monitoring tools and libraries
To monitor the system we used \emph{The Simple Logging Facade for Java} (SLF4J) in combination with \emph{Log4J}.
SLF4J provides a facade for Log4J to allow other logging backends to be configured at runtime.
We used simple logging output to measure the performance and other statistics of the system.
%other tools

\subsection{Experiments}
%Desribe the experiments per system feature
%Format: describe workload, present operation, analyze result.
\subsubsection{Provisioning}

\subsubsection{Allocation}

\subsubsection{Reliability}

\subsubsection{Monitoring components}

%Report charged-time
%Report charged cost
%Report service metrics

\section{Discussion}

\section{Conclusion}

\appendix
\section{Time Sheets}
\end{document}
